\begin{tabular}{lllll}
ID & Requirement & ParentRequirement & ChildRequirement & VerificationMethod \\ 
\hline 
AUT 1.1 & All motion planning shall include horizontal plane bounds on UAS movement. & AUT 1 &  & A \\ 
AUT 2.1 & All motion planning shall include vertical axis bounds on UAS movement. & AUT 2 &  & A \\ 
AUT 3.1 & UAS shall recognize when a height of 30-60 ft relative to the ground is reached during the takeoff phase and will begin searching for RGVs. & AUT 3 &  & D \\ 
AUT 3.2 & UAS shall recognize when one RGV has been localized and will begin searching for any additional RGV. & AUT 3 &  & D \\ 
AUT 3.3 & UAS shall default to the search phase when no RGV is detected or when a UI overload control is deactivated. & AUT 3 &  & D \\ 
AUT 4.1 & UAS shall recognize when an initial RGV has been detected and will begin trailing the motion of the RGV. & AUT 4 &  & D \\ 
AUT 4.2 & UAS shall recognize when an additional RGV has been detected and will begin trailing the motion of the additional RGV, taking into account not to follow any prior localized RGVs. & AUT 4 &  & D \\ 
AUT 5.1 & UAS shall recognize when a tracked RGV stops its motion, characterized by a speed of less than TBD m/s, and will then begin coarse localization. & AUT 5 &  & D \\ 
AUT 6.1 & UAS shall recognize when an RGV has been coarsely localized for at least 60 seconds and will then begin fine localization. & AUT 6 &  & D \\ 
AUT 7.1 & UAS shall recognize when all RGVs have been finely localized for at least 30 seconds and will then begin joint localization on all RGVs. & AUT 7 &  & D \\ 
AUT 8.1 & UAS shall recognize when all RGVs have been jointly localized for at least 20 seconds and will register as mission complete. & AUT 8 &  & D \\ 
AUT 8.2 & UAS shall perform a celebratory maneuver upon successful completion. & AUT 8 &  & D \\ 
AUT 8.3 & UAS shall initiate the landing phase after mission completion. & AUT 8 &  & D \\ 
AUT 9.1 & UAS shall perform a designated search pattern to attempt to identify RGVs. & AUT 9 &  & D \\ 
AUT 9.2 & UAS shall perform the search pattern at a rate that ensures accurate sensing. & AUT 9 &  & A \\ 
AUT 10.1 & UAS shall always be less than 40 ft horizontal ground distance away from any RGV it is currently trailing. & AUT 10 &  & A \\ 
AUT 10.2 & UAS shall be able to follow RGV motion at speeds of up to 2 m/s. & AUT 10 &  & A \\ 
AUT 11.1 & UAS shall perform an orbiting flight path maneuver around the RGV it is coarse localizing. & AUT 11 &  & A \\ 
AUT 11.2 & UAS shall perform the orbiting flight path maneuver at a rate and path that informs accurate sensing. & AUT 11 &  & A \\ 
AUT 12.1 & UAS shall perform an informative flight path maneuver that will reduce RGV location error during fine localization by TBD amount. & AUT 12 &  & A \\ 
AUT 12.2 & UAS shall perform the informative flight path maneuver at a rate and path that informs accurate sensing. & AUT 12 &  & A \\ 
AUT 13.1 & UAS shall perform a joint localization flight path maneuver that will allow the capture of both RGVs by either the primary or secondary sensor. & AUT 13 &  & A \\ 
AUT 13.2 & UAS shall perform the joint localization flight path maneuver at a rate that ensures accurate sensing. & AUT 13 &  & A \\ 
AUT 14.1 & UI shall display a 2D map of the 150 ft x 150 ft environment. & AUT 14 &  & T \\ 
AUT 14.2 & UI shall display the current drone UAS location relative to the environment. & AUT 14 &  & T \\ 
AUT 14.3 & UI shall display an estimated state for acquired RGV location(s) at each stage of localization. & AUT 14 &  & T \\ 
AUT 14.4 & UI shall display the current phase of operation. & AUT 14 &  & T \\ 
AUT 14.5 & UI shall display a timer indicating the amount of time spent in the current phase. & AUT 14 &  & T \\ 
AUT 15.1 & UI shall have the ability to initiate the UAS mission, beginning the startup sequence. & AUT 15 &  & T \\ 
AUT 15.2 & UI shall have the ability to activate a hover mode for the UAS, stopping all motion and mission operations until deactivated. & AUT 15 &  & T \\ 
AUT 15.3 & UI shall have the ability to abort the UAS mission, commanding the UAS to return to base and land. & AUT 15 &  & T \\ 
AUT 15.4 & UI shall be able to switch the UAS control method to a manual handheld controller, deactivating the autonomous controller. & AUT 15 &  & T \\ 
AUT 16.1 & All UAS motion shall be limited to speeds of less than TBD m/s. & AUT 16 &  & I \\ 
AUT 16.2 & All UAS motion shall be limited to accelerations of less than TBD g’s. & AUT 16 &  & I \\ 
AUT 16.3 & Any UAS motion detected outside the defined mission area, besides takeoff and landing, shall be immediately aborted, and the UAS shall be set to hover mode. & AUT 16 &  & I \\ 
AUT 16.4 & UAS autonomous flight initiation shall occur after a TBD second countdown to ensure safe initiation. & AUT 16 &  & I \\ 
UAS 1.1 & The UAS shall provide adequate structural support and impact absorption to provide protection for hardware and bystanders in case of accident. & UAS 1 &  & T/A \\ 
UAS 1.2 & The UAS shall be equipped with proper mounting solutions for sensors. & UAS 1, UAS 6 &  & I/T \\ 
UAS 2.1 & UAS shall have a battery that is capable of safely providing power to all systems throughout the entire flight. & UAS 2 &  & T/A \\ 
UAS 2.2 & UAS shall have access to a computer that can adequately handle all computational requirements for drone 
operation. & UAS 2 &  & T \\ 
UAS 4.1 & Systems and motors shall be calibrated to provide specific movement given an input. & UAS 4, UAS 5 &  & T \\ 
UAS 7.1 & UI design shall allow for sending and recieving of signals to and from the UAS. & UAS 7 &  & T \\ 
UAS 7.2 & UAS shall be able to reliably send and recieve all signals over any distance within the operational boundaries. & UAS 7 &  & T \\ 
UAS 9.1 & Motors shall be strong enough to allow the drone to move at speeds greater than the movement speed of the RGV. & UAS 9 &  & T/D \\ 
UAS 10.1 & UAS shall weigh no more than 55 pounds in order to comply with FAA regulations & UAS 10 &  & I \\ 
UAS 10.2 & UAS shall be able to abort operation & UAS 10 &  & T \\ 
UAS 10.3 & UAS shall have landing gear & UAS 10 &  & I \\ 
UAS 10.4 & UAS propellers shall not interfere with each other or with the aircraft body & UAS 10 &  & D \\ 
UAS 10.5 & UAS vibrations shall be within 0.005 inches of it's center of gravity at the propellers & UAS 10 &  & T/D \\ 
UAS 10.6 & UAS shall have a diverse array of antennas & UAS 10 &  & I \\ 
UAS 10.7 & Ground station shall have a diverse array of antennas. & UAS 10 &  & I \\ 
UAS 10.8 & UAS shall have a low presence of fibrous materials & UAS 10 &  & A \\ 
UAS 10.9 & UAS shall be able to remain stable while in up to 10mph winds & UAS 10 &  & T/A \\ 
UAS 10.10 & UAS shall be stable about all axes during operation & UAS 6, UAS 10 &  & A \\ 
SNS 1.1 & All data collected during the mission shall be saved and made available for post processing. & SNS 1 &  & T \\ 
SNS 1.2 & Logged data shall be referenced to the aircraft’s state at the time of recording. & SNS 1 &  & T \\ 
SNS 1.3 & An onboard communications link shall transmit data to the ground station during flight. & SNS 1 &  & T \\ 
SNS 2.1 & The primary sensor shall be used exclusively during coarse localization. & SNS 2 &  & I \\ 
SNS 2.2 & The primary sensor shall not use additional sensing capabilities other than RGB. & SNS 2 &  & I \\ 
SNS 2.3 & The primary sensor shall not directly communicate with a given RGV. & SNS 2 &  & I \\ 
SNS 3.1 & The secondary sensor shall be used exclusively during fine localization. & SNS 3 &  & I \\ 
SNS 3.2 & The secondary sensor shall not directly communicate with a given RGV. & SNS 3 &  & I \\ 
SNS 4.1 & Each RGV shall be identified during the course of the mission. & SNS 4 &  & D \\ 
SNS 5.1 & The primary sensor shall provide state estimates of the RGV during trailing. & SNS 5 &  & D \\ 
SNS 5.2 & Each RGV shall be trailed during the course of the mission. & SNS 5 &  & D \\ 
SNS 6.1 & The UAS shall complete coarse localization within 60 seconds. & SNS 6 &  & D \\ 
SNS 6.2 & Coarse localization sensing data shall be acquired exclusively by the primary sensor. & SNS 6 &  & I \\ 
SNS 6.3 & The primary sensor shall record data at multiple aerial locations along an orbiting flight path. & SNS 6 &  & T \\ 
SNS 6.4 & Once the RGV is coarsely localized, the inertial estimation of the RGV shall be accurate to within a 5 meter radius. & SNS 6 &  & D \\ 
SNS 6.5 & Passive or active emitters mounted to the RGV shall not be used for coarse localization. & SNS 6 &  & I \\ 
SNS 6.6 & Each RGV shall be coarsely localized during the course of the mission. & SNS 6 &  & D \\ 
SNS 7.1 & The UAS shall complete fine localization within 30 seconds. & SNS 7 &  & D \\ 
SNS 7.2 & Fine localization sensing data shall be acquired exclusively by the secondary sensor & SNS 7 &  & I \\ 
SNS 7.3 & The secondary sensor shall record data at multiple locations along an informed flight path. & SNS 7 &  & D \\ 
SNS 7.4 & Fine localization shall provide an increase in accuracy for the RGVs inertial estimation to within 1 meter. & SNS 7 &  & D \\ 
SNS 7.5 & Each RGV shall be finely localized during the course of the mission. & SNS 7 &  & D \\ 
SNS 8.1 & The UAS shall complete joint localization of both RGVs within 20 seconds. & SNS 8 &  & D \\ 
SNS 8.2 & Joint localization shall provide an inertial estimate of both RGVs to within a 2 meter radius. & SNS 8 &  & T \\ 
SNS 8.3 & The sensor used for joint localization shall have a field of view that encompasses the entire mission area in the frame from a height of 60 ft. (~121°). & SNS 8 &  & I \\ 
SNS 9.1 & All radio frequency transmission shall have a frequency within a permissible bandwidth as defined by the FCC (2.4 or 5 GHz). & SNS 9 &  & I \\ 
\hline 
\end{tabular}